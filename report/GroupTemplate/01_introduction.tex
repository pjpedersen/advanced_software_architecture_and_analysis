Designing software for manufacturing systems introduces stringent requirements that strongly influence architectural decisions\cite{Wollschlaeger2017}. This paper explores how to approach such design, including requirement analysis, use cases, architectural considerations, technology trade-offs, and their impact on implementation. We also present lessons learned from evaluating the resulting prototype.  \\

Our case study focuses on an Industry 4.0 production line for customizable IoT weather displays, featuring options such as color selection and engraving. These capabilities, combined with requirements for serialization, secure provisioning, genealogy tracking, and lifecycle software support, impose strict constraints: devices must never lose traceability after assembly, quality control must ensure functional correctness, and each product must be linked to its customer. \\

We define the research question and approach in Section \ref{sec:problem}, detail the case and requirements in Section \ref{sec:use_case}, and apply software architecture methodologies to model the system. Critical components undergo formal verification, and a prototype implementation validates the design. Section \ref{sec:evaluation} presents the evaluation and key findings.