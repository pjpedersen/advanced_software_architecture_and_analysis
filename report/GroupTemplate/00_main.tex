\documentclass[conference]{IEEEtran}
\usepackage[utf8]{inputenc}
\usepackage[english]{babel}
\usepackage{icomma}
\usepackage[T1]{fontenc}

% Packages
\usepackage{cite}
\usepackage{amsmath,amssymb,amsfonts}
\usepackage{algorithmic}
\usepackage{graphicx}
\usepackage{textcomp}
\usepackage{xcolor}
\usepackage{multirow}
\usepackage{rotating}
\usepackage{mdframed}
\usepackage{hyperref}
\usepackage{tikz}
\usepackage{makecell}
\usepackage{tcolorbox}
\usepackage{amsthm}
\usepackage{pifont}
\usepackage{listings}
\lstset
{ 
    basicstyle=\footnotesize,
    numbers=left,
    stepnumber=1,
    xleftmargin=5.0ex,
}

% SCJ
\usepackage{array, multirow}
\usepackage{enumitem}
\def\BibTeX{{\rm B\kern-.05em{\sc i\kern-.025em b}\kern-.08em
    T\kern-.1667em\lower.7ex\hbox{E}\kern-.125emX}}

\begin{document}

\title{Group Report Template}

\author{
    \IEEEauthorblockN{
        Student 1\IEEEauthorrefmark{1},
        Student 2\IEEEauthorrefmark{1},
        Student 3\IEEEauthorrefmark{1},
        Student 4\IEEEauthorrefmark{1},
        Student 5\IEEEauthorrefmark{1},
    }
    \IEEEauthorblockA{
        University of Southern Denmark, SDU Software Engineering, Odense, Denmark \\
        Email: \IEEEauthorrefmark{1} \textnormal{\{student1,student2,student3,student4,student5\}}@student.sdu.dk
    }
}

\maketitle

\IEEEpubidadjcol
\begin{abstract}
This report presents an architectural prototype for an Industry 4.0 production line manufacturing customizable IoT weather displays. The system integrates heterogeneous devices—PLCs, robots, engravers, sensors—under strict requirements: 24/7 availability, continuous deployment, interoperability, and traceability. \\

Our contributions are threefold: (1) a hybrid architecture combining event-driven messaging and microservices to enable real-time coordination in a heterogeneous environment; (2) application of architectural tactics to achieve quality attributes such as availability, security, and deployability in a multi-technology stack; and (3) formal verification of safety interlocks, scheduling invariants, and liveness properties to ensure correctness during dynamic production changes. \\

The prototype demonstrates feasibility for changeover optimization, genealogy tracking, and external API integration, validating that the architecture meets functional and quality goals in a continuously deployable Industry 4.0 context.

\end{abstract}

\begin{IEEEkeywords}
Industry 4.0, software architecture, event-driven systems, microservices, formal verification.
\end{IEEEkeywords}


\section{Introduction and Motivation}
The structure of the paper is as follows. 
Section \ref{sec:problem} outlines the research question and the research approach. 
%to analyze the research question and evaluate our results.
Section \ref{sec:related_work} describes similar work in the field and how our contribution fits the field.
Section \ref{sec:use_case} presents a production reconfiguration use case.
The use case serves as input to specify a reconfigurability QA requirement in Section \ref{sec:qas}.
% Section \ref{sec:middleware_architecture} introduces the proposed reconfigurable middleware software architecture design.
Section \ref{sec:evaluation} evaluates the proposed middleware on realistic equipment in the I4.0 lab and analyzes the results against the stated QA requirement.   


\section{Problem and Approach}
\label{sec:problem}
\emph{Problem.}

\emph{Research questions:}
\begin{enumerate}
    \item  
    \item 
\end{enumerate}

\emph{Approach.}
The following steps are taken to answer this paper's research questions: 
\begin{enumerate}
    \item 
\end{enumerate}

\section{Related work}
\label{sec:related_work}
This Section addresses existing contributions by examining xxx in the I4.0 domain. 
In total, x papers are investigated. 

In \cite{MyRef}, experiences are elaborated on a three-layer architecture of a reconfigurable smart factory for drug packing in healthcare I4.0. 

% The paper \cite{Yazen2010Ontology} proposes an ontology agent-based architecture for inferring  new configurations to adapt to changes in manufacturing requirements and/or environment.

% In \cite{Leitao2016Specification,Angione2017Integration} an architecture for a reconfigurable production system is specified.
% Two objectives for reconfiguration and how they can be reached are described.

% Several papers \cite{Koren1999Reconfigurable,Koren2010Design,Bortolini2018Reconfigurable} describe reconfigurable manufacturing systems that are cost-effective and responsive to market changes.

% All contributions provide valuable knowledge about reconfiguration but lack a study of the software architecture perspective that specifies a quantifiable reconfigurability architectural requirement, a software architecture that adopts the architectural requirements, and evaluates the architectural requirement. 

\section{Use Case}
\label{sec:use_case}
This Section introduces the use cases.


\section{Quality Attribute Scenario}
\label{sec:qas}

This Section introduces the specified x QASes.
The QASes are developed based on the use case.

% Description of the overall architecture designs
% Argue for tactics used to archieve the QASes
% Discuss the trade-offs

\section{Design and Analysis Modeling}
\label{sec:design_and_analysis_modeling}
Design and analysis modelling.


\section{Formal verification and validation}
\label{sec:formal_v_and_v}
Formal verification and validation of system(s).

\section{Evaluation}
\label{sec:evaluation}
% Empirical evaluation
This Section describes the evaluation of the proposed design.
Section \ref{sec:design} introduces the design of the experiment to evaluate the system. 
Section \ref{sec:measurements} identifies the measurements in the system for the experiment.
Section \ref{sec:pilot_test} describes the pilot test used to compute the number of replication in the actual evaluation. 
Section \ref{sec:analysis} presents the analysis of the results from the experiment. 

\subsection{Experiment design}
\label{sec:design}

\subsection{Measurements}
\label{sec:measurements}

\subsection{Pilot test}
\label{sec:pilot_test}

\subsection{Analysis}
\label{sec:analysis}



\section{Conclusion}
\label{sec:conclusion}
Conclusion of the report, discussion and relevant future work.

\subsection{Discussion}
\label{sec:Discussion}

\subsection{Future work}
\label{sec:future_work}

\bibliographystyle{IEEEtran}
\bibliography{references}

\vspace{12pt}

\newpage
\section*{Contributions}
\begin{tabular}{l|p{0.6\linewidth}}
Name & Contribution\\
\end{tabular}
\end{document}
